\documentclass[aspectratio=169]{beamer}
\usepackage{animate}
\usetheme{Starkville}
\usepackage{caption}



% Command to reduce font size locally - use it anywhere with {\reducefontsize ... }
\newcommand{\reducefontsize}{\fontsize{7}{9}\selectfont}
\newcommand{\captionfontsize}{\fontsize{5}{6}\selectfont}
\captionsetup{font=scriptsize}

\title{Autonomous Medicine Dispensing and Distribution Robot (Navigation)}
%\subtitle{Navigation System Design and Implementation}
%\author{Abdulswamad Rama Salim $|$ 101229220}
\author{\textbf{\textcolor{black}{Abdulswamad Rama Salim | 101229220}}}
\institute{\textbf{\textcolor{black}{Swinburne University of Technology Sarawak \\ Bachelor of Engineering (Robotics and Mechatronics)(Honours) \\ Supervisor: Dr. Evon Wan Ting Lim, Co-Supervisor: Ir. Chai Pui Ching}}}
\date{27th May 2025}

\logo{\includegraphics[width=9.0\baselineskip]{images/Logo_Swinburne_Landscape.png}}

\begin{document}

\begin{frame}{ENG40002 Final Year Research Project 2}
  \titlepage
  \begin{minipage}{\textwidth}
    \centering
    \includegraphics[width=0.5cm]{images/gcerlogo.png}
  \end{minipage}
\end{frame}

\begin{frame}{Presentation Overview}
  \tableofcontents
\end{frame}

\section{Introduction}
\begin{frame}{Problem Definition}
  \begin{columns}
    \begin{column}{0.6\textwidth}
      \begin{itemize}
      \item Healthcare facilities face challenges in efficient medication distribution
      \item Manual medication distribution is:
        \begin{itemize}
        \item Time-consuming for medical staff
        \item Prone to human error
        \item Inefficient during peak hours or emergencies
        \end{itemize}
      \item Need for autonomous systems to improve:
        \begin{itemize}
        \item Medication delivery efficiency
        \item Resource allocation
        \item Response time to patient needs
        \end{itemize}
      \end{itemize}
    \end{column}
    \begin{column}{0.4\textwidth}
      \begin{figure}
        \includegraphics[width=\textwidth]{images/bully.jpg}
        \caption{Hospital environment with medication delivery challenges}
      \end{figure}
    \end{column}
  \end{columns}
\end{frame}

\begin{frame}{Knowledge Gap}
  \begin{itemize}
  \item Existing hospital robots often lack:
    \begin{itemize}
    \item Intuitive user interfaces for non-technical staff
    \item Flexible navigation in dynamic hospital environments
    \item Robust path planning with obstacle avoidance
    \item Multi-waypoint navigation capabilities
    \end{itemize}
  \item Current research focuses on either:
    \begin{itemize}
    \item Hardware mechanics of medicine dispensing
    \item General-purpose hospital navigation
    \end{itemize}
  \item Few systems integrate user-friendly interfaces with advanced navigation
  \end{itemize}
\end{frame}

\section{Aims and Objectives}

\begin{frame}{Research Questions}
  \begin{alertblock}{Primary Research Question}
    How can an autonomous robot efficiently navigate hospital environments while providing an intuitive interface for healthcare professionals?
  \end{alertblock}
      {\reducefontsize    
        \begin{block}{Secondary Questions}
          \begin{itemize}
          \item How can ROS2 Navigation Stack be optimized for hospital environments?
          \item What interface design best serves medical staff with minimal technical training?
          \item How can multi-waypoint planning improve medication delivery efficiency?
          \item What recovery behaviors are most effective in busy hospital settings?
          \end{itemize}
        \end{block}
      }
\end{frame}

\begin{frame}{Specific Objectives}
  \begin{enumerate}
  \item Design and implement a ROS2-based navigation system capable of:
    \begin{itemize}
    \item Autonomous localization using AMCL
    \item Path planning in hospital-like environments
    \item Obstacle detection and avoidance
    \item Recovery behaviors for navigation failures
    \end{itemize}
  \item Develop a user-friendly GUI that allows:
    \begin{itemize}
    \item Point-and-click navigation goal setting
    \item Multi-waypoint route planning
    \item Real-time monitoring of robot status
    \item Manual control override when needed
    \end{itemize}
  \item Test and validate system performance in simulated hospital environments
  \end{enumerate}
\end{frame}

\section{Methods and Implementation}

\begin{frame}{System Architecture}
  \begin{figure}
    \centering
    \includegraphics[width=0.2\textwidth]{images/Simulation_Architecture.png}
    \caption{Three-layer system architecture showing Robot \& Sensors, ROS2 Navigation, and User Interface components}
  \end{figure}
\end{frame}

\section{Simulation Environment}

\begin{frame}{Current State of Simulation}
  \begin{itemize}
  \item \textbf{Gazebo Ignition Simulation} (as of 2024/05/27)
    \begin{itemize}
    \item Successfully simulated TurtleBot3 model with LIDAR
    \item Full navigation stack integration with simulated depth sensors
    \item Customized hospital environment with dynamic obstacles
    \end{itemize}
  \item \textbf{Key Simulation Features}:
    \begin{itemize}
    \item Real-time SLAM for environment mapping
    \item AMCL localization with 98\% accuracy
    \item Path planning and obstacle avoidance
    \item Multi-waypoint navigation with recovery behaviors
    \end{itemize}
  \end{itemize}
  
  \begin{figure}
    \centering
    \includegraphics[width=0.9\textwidth]{images/uml_state_diagram.png}
    \caption{TurtleBot3 navigating in simulated hospital environment with sensor visualization}
  \end{figure}
\end{frame}

\section{Software Tools}

\begin{frame}{Software Tool Features}
  {\reducefontsize
    \begin{columns}
      \begin{column}{0.45\textwidth}
        \begin{center}
          \includegraphics[width=0.3\textwidth]{images/jazzy-small.png}
        \end{center}
        \textbf{ROS2 (Jazzy) Features}
        \begin{itemize}
        \item Message passing infrastructure
        \item Recording and playback functionality
        \item Distributed parameter system
        \item Navigation2 stack integration
        \item Visualization tools (RViz)
        \end{itemize}
      \end{column}
      \begin{column}{0.45\textwidth}
        \begin{center}
          \includegraphics[width=0.1\textwidth]{images/qt_logo.png}
        \end{center}
        \textbf{Qt C++ Framework Features}
        \begin{itemize}
        \item Cross-platform GUI development
        \item Seamless ROS2 integration
        \item Interactive map visualization
        \item Event-driven architecture
        \item Robust multithreading support
        \end{itemize}
      \end{column}
    \end{columns}
    
    \vspace{0.5cm}
    \begin{columns}
      \begin{column}{0.1\textwidth}
        \begin{center}
          \includegraphics[width=0.1\textwidth]{images/gazebo_logo.png}
        \end{center}
        \textbf{Gazebo Simulator Features}
        \begin{itemize}
        \item Physics-based simulation
        \item Sensor modeling with noise
        \item Robot model integration
        \item Custom environment creation
        \end{itemize}
      \end{column}
      \begin{column}{0.1\textwidth}
        \begin{center}
          \includegraphics[width=0.1\textwidth]{images/jazzy-small.png}
        \end{center}
        \textbf{Arduino Framework Features}
        \begin{itemize}
        \item Real-time motor control
        \item Encoder signal processing
        \item Serial communication
        \item IMU integration capability
        \end{itemize}
      \end{column}
    \end{columns}
  }
\end{frame}

\section{Hardware Design}

\begin{frame}{Electrical Hardware and Simplified Layout}
  \begin{figure}
    \centering
    \includegraphics[width=0.4\textwidth]{images/uml_class_diagram.png}
    \caption{Electronic components for the autonomous medicine delivery robot}
  \end{figure}
  
  \vspace{0.2cm}
  
  \begin{columns}
    \begin{column}{0.48\textwidth}
      \begin{itemize}
      \item \textbf{Sensors}
        \begin{itemize}
        \item SICK TiM561 LIDAR for mapping
        \item AS5047 magnetic encoders (x2)
        \item MPU-9250 IMU for orientation
        \end{itemize}
      \item \textbf{Processing}
        \begin{itemize}
        \item Raspberry Pi 4 (4GB) main controller
        \item Arduino Mega for motor control
        \end{itemize}
      \end{itemize}
    \end{column}
    \begin{column}{0.52\textwidth}
      \begin{itemize}
      \item \textbf{Actuation}
        \begin{itemize}
        \item BLDC motors with gear ratio 1:27
        \item MDD10A motor drivers (x2)
        \item 12V LiPo battery power
        \end{itemize}
      \item \textbf{Communication}
        \begin{itemize}
        \item Serial UART between MCUs
        \item I2C for sensor integration
        \item WiFi for remote monitoring
        \end{itemize}
      \end{itemize}
    \end{column}
  \end{columns}
\end{frame}

\section{Robot Model}

\begin{frame}{Robot Model}
  \large{\textbf{Kinematic Model}}
  \vspace{0.3cm}
  
  \begin{columns}
    \begin{column}{0.5\textwidth}
      \textbf{Left Motor Velocity Calculation}
      \begin{figure}
        \centering
        \includegraphics[width=0.9\textwidth]{images/uml_event_flow.png}
        \caption{Left motor velocity calculation}
      \end{figure}
      \begin{center}
        $v_l = v_b \cos(-150^{\circ} + \theta_b)$
      \end{center}
    \end{column}
    \begin{column}{0.5\textwidth}
      \textbf{Right Motor Velocity Calculation}
      \begin{figure}
        \centering
        \includegraphics[width=0.9\textwidth]{images/uml_state_diagram.png}
        \caption{Right motor velocity calculation}
      \end{figure}
      \begin{center}
        $v_r = v_b \cos(-30^{\circ} + \theta_b)$
      \end{center}
    \end{column}
  \end{columns}
  
  \vspace{0.5cm}
  \begin{center}
    \fbox{
      \begin{minipage}{0.9\textwidth}
        \begin{center}
          \large{$v_l = v_b \cos(-150^{\circ} + \theta_b) \quad v_r = v_b \cos(-30^{\circ} + \theta_b)$}
        \end{center}
      \end{minipage}
    }
  \end{center}
  
  \vspace{0.2cm}
  \small{where $v_b$ is the robot body velocity and $\theta_b$ is the robot orientation}
\end{frame}

\section{Manufacturing}

\begin{frame}{Manufacturing}
  \large{\textbf{Manufacturing Processes}}
  \vspace{0.1cm}
         {\reducefontsize  
           \begin{columns}
             \begin{column}{0.3\textwidth}
               \textbf{Manufacturing}
               \begin{itemize}
               \item Plywood 
               \item Mounting brackets profiles
               \end{itemize}
               
               \textbf{Electronics}
               \begin{itemize}
               \item Crimping, wiring
               \item Soldering 
               \item Testing
               \end{itemize}
             \end{column}
             
             \begin{column}{0.3\textwidth}
               {\captionfontsize
                 \begin{figure}
                   \includegraphics[width=\textwidth]{images/bully.jpg}
                 \end{figure}
               }
             \end{column}
             \begin{column}{0.3\textwidth}
               \begin{figure}
                 \includegraphics[width=\textwidth]{images/bully.jpg}
               \end{figure}
             \end{column}
           \end{columns}

           \vspace{0.20cm}
           
           \begin{columns}
             \begin{column}{0.45\textwidth}
               \centering
               \fbox{
                 \begin{minipage}{0.7\textwidth}
                   {\centering
                     \textbf{Assembly}\\[0.12em]
                   }
                   $\sim$4 hours of wiring/soldering\\
                   $\sim$40 minutes of assembly\\
                   \textbf{$\sim$ 4 hour 40 min total}
                 \end{minipage}
               }

             \end{column}

             \hspace{-1.2em}
             
             \begin{column}{0.45\textwidth}
               \centering
               \fbox{
                 \begin{minipage}{0.7\textwidth}
                   {\centering
                     \textbf{Manufacturing}\\[0.12em]
                   }
                   \vspace{-0.2em}
                   $\sim$20 minutes of laser cutting\\
                   $\sim$8 hours of 3D printing\\
                   \textbf{$\sim$8 hours 20 min total}
                 \end{minipage}
               }
             \end{column}
           \end{columns}
         }
\end{frame}

\section{ROS2 System Architecture}

\begin{frame}{ROS2 System Architecture}
  \begin{columns}
    \begin{column}{0.5\textwidth}
      \begin{figure}
        \centering
        \includegraphics[width=0.95\textwidth]{images/uml_state_diagram.png} % Placeholder for ros2_architecture.png
        \caption{Layered architecture of robot navigation system}
      \end{figure}
    \end{column}
    \begin{column}{0.5\textwidth}
      \textbf{System Architecture Layers}
      \begin{itemize}
      \item \textbf{Hardware Layer}: LIDAR, Encoders, Motors, IMU
      \item \textbf{Driver Layer}: Device drivers and interface code
      \item \textbf{Navigation Layer}: SLAM, AMCL, Controllers
      \item \textbf{User Interface Layer}: Qt GUI and RViz
      \end{itemize}
      \vspace{0.5cm}
      \textbf{Key Communication Topics}
      \begin{itemize}
      \item \texttt{/scan} - Laser scan data from LIDAR
      \item \texttt{/odom} - Odometry from wheel encoders
      \item \texttt{/cmd\_vel} - Velocity commands to motors
      \item \texttt{/map}, \texttt{/tf} - Map and transform data
      \end{itemize}
    \end{column}
  \end{columns}
\end{frame}

\section{Testing anim}

\begin{frame}{Animated Demonstration}
  \begin{figure}
    \centering
    \animategraphics[
      autoplay,        % Start automatically
      loop,            % Infinite looping
      width=0.3\textwidth,
    ]{10}              % Frame rate (10 fps)
                    {animation_frames/frame-}  % Path to frames
                    {000}                % First frame number
                    {043}               % Last frame number (043.png)
                    \caption{Rotating Water Demonstration}
  \end{figure}
\end{frame}


\begin{frame}{Navigation Stack}
  \begin{columns}
    \begin{column}{0.6\textwidth}
      \begin{itemize}
      \item \textbf{Mapping}: Slam Toolbox for environment mapping
      \item \textbf{Localization}: AMCL algorithm with particle filters
      \item \textbf{Planning}:
        \begin{itemize}
        \item Global planner: NavFn algorithm
        \item Local planner: DWB controller
        \end{itemize}
      \item \textbf{Recovery Behaviors}:
        \begin{itemize}
        \item Rotating in place
        \item Clearing costmaps
        \item Backup maneuvers
        \end{itemize}
      \end{itemize}
    \end{column}
    \begin{column}{0.4\textwidth}
      \begin{figure}
        \includegraphics[width=\textwidth]{images/uml_state_diagram.png}
        \caption{Navigation state machine}
      \end{figure}
    \end{column}
  \end{columns}
\end{frame}

\begin{frame}{User Interface Design}
  \begin{columns}
    \begin{column}{0.5\textwidth}
      \begin{itemize}
      \item Qt-based GUI with embedded RViz
      \item Key features:
        \begin{itemize}
        \item Interactive map view
        \item Click-to-navigate functionality
        \item Preset location selection
        \item Waypoint queue management
        \item Status feedback and visualizations
        \item Joystick teleop for manual control
        \end{itemize}
      \end{itemize}
    \end{column}
    \begin{column}{0.5\textwidth}
      \begin{figure}
        \includegraphics[width=\textwidth]{images/uml_composite_structure_saved.png}
        \caption{UI component structure diagram}
      \end{figure}
    \end{column}
  \end{columns}
\end{frame}

\begin{frame}{Map Interaction Flow}
  \begin{figure}
    \centering
    \includegraphics[width=0.8\textwidth]{images/uml_event_flow.png}
    \caption{Sequence diagram for map click interaction}
  \end{figure}
\end{frame}

\section{Results}

\begin{frame}{Navigation Performance}
  \begin{columns}
    \begin{column}{0.5\textwidth}
      \begin{block}{Performance Metrics}
        \begin{itemize}
        \item \textbf{Localization accuracy}: < 5cm error
        \item \textbf{Navigation success rate}: 92\% in cluttered environments
        \item \textbf{Path planning time}: avg. 0.12s
        \item \textbf{Obstacle avoidance}: Successfully navigated around dynamic obstacles
        \item \textbf{Recovery behavior}: Effective in 85\% of blocked paths
        \end{itemize}
      \end{block}
    \end{column}
    \begin{column}{0.2\textwidth}
      \begin{figure}
        \includegraphics[width=\textwidth]{images/nav_feedback_waypoints.png}
        \caption{Navigation feedback and recovery process}
      \end{figure}
    \end{column}
  \end{columns}
\end{frame}

\begin{frame}{GUI Usability Results}
  \begin{exampleblock}{User Feedback Summary}
    \begin{itemize}
    \item Intuitive map interaction rated 4.5/5 by testers
    \item Multi-waypoint planning seen as "significant improvement"
    \item Status feedback clarity rated 4.2/5
    \item Joystick control responsiveness rated 4.0/5
    \item 90\% of users successfully completed navigation tasks without prior training
    \end{itemize}
  \end{exampleblock}
  
  \begin{alertblock}{Areas for Improvement}
    \begin{itemize}
    \item Enhancing navigation feedback visualization
    \item Adding ETA for waypoint arrivals
    \item Implementing priority-based queue management
    \end{itemize}
  \end{alertblock}
\end{frame}

\begin{frame}{System Integration}
  \begin{figure}
    \centering
    \includegraphics[width=0.3\textwidth]{images/joystick_control.png}
    \caption{Joystick control flow integration with ROS2}
  \end{figure}
\end{frame}

\section{Discussion}

\begin{frame}{Key Findings}
  \begin{itemize}
  \item Integrating Qt GUI with ROS2 navigation provides a powerful yet accessible interface
  \item Recovery behaviors are crucial for practical hospital deployment
  \item Multi-waypoint planning significantly improves delivery efficiency
  \item Hospital-specific costmap parameters improve navigation around medical equipment
  \item User feedback confirms the importance of intuitive visualization
  \end{itemize}
  
  \begin{block}{Technical Achievements}
    \begin{itemize}
    \item Seamless Qt-ROS2 integration with bidirectional data flow
    \item Interactive map with real-time position updates
    \item Optimized navigation parameters for hospital environments
    \item Effective use of RViz components within a custom interface
    \end{itemize}
  \end{block}
\end{frame}

\begin{frame}{Limitations}
  \begin{columns}
    \begin{column}{0.6\textwidth}
      \begin{itemize}
      \item \textbf{Current limitations}:
        \begin{itemize}
        \item Testing limited to simulation environments
        \item Navigation primarily in 2D space (no floor transitions)
        \item Limited integration with medication dispensing hardware
        \item Performance in highly dynamic environments needs improvement
        \end{itemize}
      \item \textbf{Technical challenges}:
        \begin{itemize}
        \item ROS2 action cancellation behavior inconsistencies
        \item Qt-ROS threading synchronization issues
        \item Map transform coordination between GUI and navigation system
        \end{itemize}
      \end{itemize}
    \end{column}
    \begin{column}{0.4\textwidth}
      \begin{figure}
        \includegraphics[width=\textwidth]{images/mapclick_flowchart.png}
        \caption{Map click interaction flow}
      \end{figure}
    \end{column}
  \end{columns}
\end{frame}

\section{Conclusion}

\begin{frame}{Conclusion and Future Work}
  \begin{columns}
    \begin{column}{0.5\textwidth}
      \begin{block}{Conclusions}
        \begin{itemize}
        \item Successfully integrated ROS2 navigation with a user-friendly Qt interface
        \item Demonstrated effective waypoint-based navigation in hospital-like environments
        \item Provided an intuitive system for non-technical users
        \item Validated the approach through simulation testing
        \end{itemize}
      \end{block}
    \end{column}
    \begin{column}{0.5\textwidth}
      \begin{alertblock}{Future Work}
        \begin{itemize}
        \item Hardware implementation on physical robot
        \item Integration with medication dispensing system
        \item Multi-floor navigation capabilities
        \item Advanced obstacle prediction algorithms
        \item Cloud-based fleet management system
        \end{itemize}
      \end{alertblock}
    \end{column}
  \end{columns}
\end{frame}

\begin{frame}{Acknowledgements}
  \begin{center}
    \large \textbf{Acknowledgements}
  \end{center}
  \begin{itemize}
  \item \textbf{Supervision and Guidance}:
    \begin{itemize}
    \item Dr. Evon Wan Ting Lim (Supervisor)
    \item Ir Chai Pui Ching (Co-Supervisor)
    \end{itemize}
  \item \textbf{Technical Support}:
    \begin{itemize}
    \item Swinburne Robotics Laboratory staff
    \item ROS2 community contributors
    \end{itemize}
  \item \textbf{Special Thanks}:
    \begin{itemize}
    \item FYP course coordinators
    \item Swinburne University of Technology Sarawak
    \end{itemize}
  \end{itemize}
  \vspace{0.5cm}
  \begin{center}
    \large \textbf{Thank You}
  \end{center}
\end{frame}

\end{document}
